%!TEX root =  ../rootOfProject.tex
\chapter{Latex} 
Hier wird \gls{latex} verwendet. Es geht hier unter anderem um \gls{ip} und auch \gls{imcp}.
Aber mehr um \gls{ip}, später \gls{imcp}. 
\paragraph{ein kurzer Paragraph}
ein noch kürzerer Beitrag
\subparagraph{sub P}

\section{Use Case-Diagramm}
\cite{sigfridsson,wordnet}
Lorem ipsum dolor sit amet, consetetur sadipscing elitr, sed diam nonumy eirmod tempor invidunt ut labore et dolore magna aliquyam erat, sed diam voluptua. \\
\\
1. \ac{USB} \\
2. \ac{USB} \\
3. \ac{IP} \ac{ICMP} \ac{NAT}
4. \ac{IP} \ac{ICMP} \ac{NAT}

Lorem ipsum dolor sit amet, consetetur sadipscing elitr, sed diam nonumy eirmod tempor invidunt ut labore et dolore magna aliquyam erat, sed diam voluptua. 

\begin{figure}[htb]
	\centering
	\includegraphics[scale=0.2,]{img/sample-logo.png}
	\caption[Dies ist der kurze Text im Abb Verzeichnis]{Dies ist der lange Text unter dem Bild, der sich über mehrere Zeilen erstrecken kann.}
	\label{fig:my_label}
\end{figure}