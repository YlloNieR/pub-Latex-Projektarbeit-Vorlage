%!TEX root = rootOfProject.tex
% wird nach Erscheinung sortiert
% Punkte für Satezende werden automatisch generiert
\makeglossaries
\newglossaryentry{imcp}
{
	name=ICMP,
	description={
		Das Internet Control Message Protocol (ICMP) dient in Rechnernetzwerken dem Austausch von Informations- und Fehlermeldungen über das Internet-Protokoll in der Version 4 (IPv4). Für IPv6 existiert ein ähnliches Protokoll mit dem Namen ICMPv6. ... ICMP-Nachrichten werden in IP-Paketen gekapselt}
}
\newglossaryentry{latex}
{
	name=latex,
	description={
		Is a markup language specially suited 
		for scientific documents asd}
}

\newglossaryentry{ip}
{
	name=ip,
	description={
		Das Internet Protocol (IP) ist ein in Computernetzen weit verbreitetes Netzwerkprotokoll und stellt durch seine Funktion die Grundlage des Internets dar. Das IP ist die Implementierung der Internetschicht des TCP/IP-Modells bzw. der Vermittlungsschicht (engl. Network Layer) des OSI-Modells}
}